% !Mode\dots ``TeX:UTF-8''
% !TEX root = ../bare_jrnl.tex


\section{Introduction}
\label{sec:intro}


\IEEEPARstart{I}n 1960s, Nobel Prize laureates Jacob and Monod~\cite{Jacob1961Genetic} found that ``Any cell contains a number of regulatory genes that act as switches and can turn one another on and off. If genes can turn one another on and off, then you can have genetic circuits.'' Inspired by these Boolean-type actions in genetic circuits, Boolean networks (\BNs) were firstly proposed by Kauffman \cite{Kauffman1968Metabolic} for modeling nonlinear and complex biological systems. 

{\BNs} are a type of discrete-time dynamical systems which can be represented as directed graphs. In a BN, each node has only two values ``0" and ``1", and they can change in different time points.  For a node $n_i$, we use $n_i(t)$ to denote the value of it at the time $t$.
In general, $n_i(t+1)$ is determined by a logical function of $n_j(t),\ldots,n_p(t)$ if  there are  edges from $n_j,\ldots,n_p$ to $n_i$.  
%The logical operators used in  the logical functions include AND, OR, NO, XOR.  
Some general descriptions of the \BNs\ and their applications to biological systems can be found in~\cite{Kauffman1968Metabolic}. There exists a large number of  systems works, both natural and artificial, which are modelled by \BNs, e.g. ~\cite{Akutsu2000Inferring, Shmulevich2002From, Faur2006Dynamical,Green2007The,Lou2010Multi}.
 

\BNs\ can be naturally extended to Boolean control networks (\BCNs) with external regulations and perturbations~\cite{Ideker2001A}. \BCNs\ have been applied to  various real-life problems. Applications can be found in 
structural and functional analysis of signaling and regulatory networks~\cite{Kaufman1999A, Klamt2006A}, 
abduction based drug target discovery~\cite{Biane2017Abduction}, 
and pursuing evasion problems in polygonal environments~\cite{Thunberg2011A}. 

 There are three kinds of nodes in \BCNs. They {\em input-nodes}, {\em state-nodes}  and {\em output-nodes}, and at  any moment of time each node takes a Boolean value.  The value of an output-node at time point $t$ (where $t\geq 0$)  depends on the values of state-nodes at time $t$, but the value of each state-node at time point $t+1$ is determined by a  Boolean function of the values of the input-nodes and state-nodes at time point $t$. However,  we can only control the  value of the input-nodes and observe those of the output-nodes. This means that we cannot observe the cahnge of the values of the state-nodes. Therefore, it is important to find a way to determine the value of the state-nods at $t$ from the values of of input nodes and the values of the output-nodes  at a sequence of time points $t,\ldots t+k$. This is in general known as the {\em observability} of a \BCN. 



Observability is one of the two basic  properties related to  the control-theoretic problems of \BCNs. The other is known as {\em controllability}. The work in ~\cite{Akutsu2007Control} shows that the problem of determining the controllability of \BCNs\ is {\bf NP}-hard. Moreover, it  points out that ``One of the major goals of systems biology is to develop a control theory for complex biological systems.''  Since then, research on  controllability and observability of \BNs\ and \BCNs\ has drawn a great attention, e.g. \cite{cheng2009controllability, Zhao2010Input, Cheng2011Identification, Cheng2011Analysis} and \cite{Fornasini2013Observability}. %There, it is further noted that the controllability and observability are the basic control-theoretic problems of \BCNs. % Among these studies, \emph{semi-tensor product} (\STP) is one of useful tools to deal with  

 The concept of observability was first proposed in~\cite{cheng2009controllability}. It is about how  to determine  the values of the  state-nodes  of a \BCN\  at time $0$ from the values of  input-nodes  and output-nodes at a sequence of time points $0,\ldots k$. Four  types of observability have been investigated in literature. Algorithms are also developed for  these types of observability, which we introduce below.

Given a \BCN\ with $m$ input-nodes $(\mathsf{i}_1,\ldots, \mathsf{i}_m)$, $n$ state-nodes $(\mathsf{s}_1,\ldots, \mathsf{s}_n)$ and $q$ output notes  $(\mathsf{o}_1,\ldots, \mathsf{o}_q)$,  we use the {\em input vector} $\mathsf{i}(t)=(\mathsf{i}_1(t)\ldots\mathsf{i}_m (t))$, {\em state vector} $\mathsf{s}(t)=(\mathsf{s}_1(t), \ldots \mathsf{s}_n(t))$ and {\em output  vector} $\mathsf{o}(t)=(\mathsf{o}_1(t) \ldots \mathsf{o}_q(t))$  to represent the values of the input-nodes, state-nodes and  output-nodes at $t$, respectively.  Then, the state vector  $\mathsf{s}(t+1)$ at time $t+1$ is determined by the input vector  $\mathsf{i}(t)$ and state vector $\mathsf{s}(t)$ at $t$  and the output  vector $\mathsf{o}(t)$ by the state vector $\mathsf{s}(t)$ as shown in Fig.~\ref{fig:10}.


 \begin{figure}[!t]
      \centering
      \framebox{\parbox{3in}{
		\centerline{\includegraphics[scale=0.17]{figures/Fig10.png}}
	}}
      
      \caption{The relationship of inputs, states and outputs.}
      \label{fig:10}
  \end{figure}
   The sequences $\mathsf{i}(0)\mathsf{i}(1)\ldots\mathsf{i}(k-1)$,  $\mathsf{s}(0)\mathsf{s}(1)\ldots\mathsf{s}(k)$, and $\mathsf{o}(0)\mathsf{o}(1)\ldots\mathsf{o}(k)$ 
 consists of several inputs, states and outputs in sequential time points,  respectively, where $k>0$. 
 Such that, for an initial state $\mathsf{s}(0)$ and a  sequence of inputs $\mathsf{i}(0)\mathsf{i}(1)\ldots\mathsf{i}(k-1)$ of a \BCN, we have its corresponding 
$\mathsf{s}(0)\mathsf{s}(1)\ldots\mathsf{s}(k)$ and $\mathsf{o}(0)\mathsf{o}(1)\ldots\mathsf{o}(k)$.  
%That is for a given  \BCN\  $\BB$,  $\mathsf{o}(0)\mathsf{o}(1)\ldots\mathsf{o}(k)$ is decide by $\mathsf{s}(0)$ and the sequence $\mathsf{i}(0)\mathsf{i}(1)\ldots\mathsf{i}(k-1)$. 
And we use $\mathsf{s}^{x}(t)$ and $\mathsf{s}^{y}(t)$ to represent different valuations of $\mathsf{s}(t)$, and similarly for input-nodes and output-nodes. 

Then the four existing observability of \BCNs\ can be described as follows. 
\begin{enumerate}
	\item The  type-1 observability was proposed in 2009 \cite{cheng2009controllability} which states that a \BCN\ $\BB$ is observable if for every $\mathsf{s}^{x}(0)$ there is an $\mathsf{i}(0)\mathsf{i}(1)\ldots\mathsf{i}(k-1)$ that can distinguish $\mathsf{s}^{x}(0)$ from other types of initial states. That is in the $\BB$, for the $\mathsf{s}^{x}(0)$ and $\mathsf{i}(0)$$\mathsf{i}(1)\ldots$$\mathsf{i}(k-1)$, for every $\mathsf{s}^{y}(0)\ne\mathsf{s}^{x}(0)$, the corresponding $\mathsf{o}^{y}(0)$$\mathsf{o}^{y}(1)\ldots$$\mathsf{o}^{y}(k)$ of $\mathsf{s}^{y}(0)$ is different from the corresponding $\mathsf{o}^{x}(0)$$\mathsf{o}^{x}(1)\ldots$$\mathsf{o}^{x}(k)$ of $\mathsf{s}^{x}(0)$. 
	%------------------------------
	\item 
	The  type-2 observability was proposed in 2010 \cite{Zhao2010Input}. It states that a \BCN\ $\BB$ is observable if for every two distinct $\mathsf{s}^{x}(0)$ and $\mathsf{s}^{y}(0)$, there is an $\mathsf{i}(0)$$\mathsf{i}(1)\ldots$$\mathsf{i}(k-1)$ that can distinguish them. That is in the $\BB$, for the $\mathsf{s}^{x}(0)$, $\mathsf{s}^{y}(0)$ and $\mathsf{i}(0)\mathsf{i}(1)\ldots\mathsf{i}(k-1)$, the corresponding $\mathsf{o}^{x}(0)\mathsf{o}^{x}(1)\ldots\mathsf{o}^{x}(k)$ of $\mathsf{s}^{x}(0)$ and the corresponding $\mathsf{o}^{y}(0)\mathsf{o}^{y}(1)\ldots\mathsf{o}^{y}(k)$ of $\mathsf{s}^{y}(0)$ are different.
	\item The  type-3 observability proposed in 2011 \cite{Cheng2011Identification}, and it states that a \BCN\ $\BB$ is observable if there is an $\mathsf{i}(0)$$\mathsf{i}(1)\ldots$$\mathsf{i}(k-1)$ that determines $\mathsf{s}(0)$. That is in the $\BB$, for the $\mathsf{i}(0)$$\mathsf{i}(1)\ldots$$\mathsf{i}(k-1)$, for every two distinct $\mathsf{s}^{x}(0)$ and $\mathsf{s}^{y}(0)$, the corresponding $\mathsf{o}^{x}(0)$$\mathsf{o}^{x}(1)\ldots$$\mathsf{o}^{x}(k)$ of $\mathsf{s}^{x}(0)$ is different from the corresponding $\mathsf{o}^{y}(0)$$\mathsf{o}^{y}(1)\ldots$$\mathsf{o}^{y}(k)$ of $\mathsf{s}^{y}(0)$.
	
	\item  The  type-4 observability proposed in 2013 \cite{Fornasini2013Observability}. It states that a \BCN\ $\BB$ is observable if every sufficient long $\mathsf{i}(0)$$\mathsf{i}(1)\ldots$$\mathsf{i}(k-1)$ can determine $\mathsf{s}(0)$. That is in the $\BB$, for every sufficient long $\mathsf{i}(0)$$\mathsf{i}(1)\ldots$ $\mathsf{i}(k-1)$, for every two distinct $\mathsf{s}^{x}(0)$ and $\mathsf{s}^{y}(0)$, the corresponding $\mathsf{o}^{x}(0)$$\mathsf{o}^{x}(1)\ldots$$\mathsf{o}^{x}(k)$ of $\mathsf{s}^{x}(0)$ is different from the corresponding $\mathsf{o}^{y}(0)$$\mathsf{o}^{y}(1)\ldots$$\mathsf{o}^{y}(k)$ of $\mathsf{s}^{y}(0)$.
\end{enumerate}
 Their formal definitions will be presented in Section~\ref{sec:pre}.

%{\color{red} (13)} 
Loosely speaking, the four types of observability can be classified into two categories: ...
In the four notions of observability, the type-1 and type-2 determine $\mathsf{s}(0)$ by running the respective procedure multiple times in parallel, and the type-3 and type-4 determine $\mathsf{s}(0)$ by taking the determining procedure once. For the details of these propositions, we refer readers to \cite{cheng2009controllability, Zhao2010Input, Cheng2011Identification,Fornasini2013Observability}. %And the second observability is the necessary and sufficient condition of determining $\mathsf{s}(0)$ by taking the determining procedure once. 

In this paper, we address the necessary and sufficient condition of determining $\mathsf{s}(0)$ by taking the determining procedure once. %It enriches the control-theory of \BCNs, it can help us determine the $\mathsf{s}(0)$ of some \BCNs\ which can not be determined before.

%\begin{problem}
%\label{pro:1}
%For a given \BCN\ $\BB$, whether its initial state $\mathsf{s}(0)$ can be determined by carrying on determining procedure once?
%\end{problem}

%Although, these four existing notions of observability can help us study some information about $\mathsf{s}(0)$ we need. 
%The problem means that whether the $\mathsf{s}(0)$ can be determined if the determining procedure can do at most once. %only third and fourth ones  can  determine the $\mathsf{s}(0)$ of \BCNs.
%Only third and fourth notions of observability can help us solve the {\em Problem~\ref{pro:1}}.
% What is more, in the third observability, a \BCN\ is observable iff there is an input sequence that determines its initial state $\mathsf{s}(0)$. Thus its condition is very strong, and the condition of fourth observability is even stronger.
%However, 
Let $\mathsf{S}(t)$ denote the set of possible valuations of $\mathsf{s}(t)$. Then naturally there is a procedure which is described as follows.
%We find that to determine $\mathsf{s}(0)$ by taking the determining procedure once we only need to complete the following procedure. 

%Informly, the procedure of our  online observability as follows. 
 %Then $\mathsf{S}(t)$ can be derived by the output sequence $\mathsf{o}(0)\mathsf{o}(1)\ldots\mathsf{o}(t)$, input sequence $\mathsf{i}(0)\mathsf{i}(1)\ldots\mathsf{i}(t-1)$ and updating rules of $\BB$.
 
 %\tl{can you make this a proper algorithm?}\gs{I have corrected this one.}
\begin{description}
	\item[Step 1] We infer $\mathsf{S}(t)$ by the $\mathsf{o}(t)$ based on the relation of $\mathsf{o}(t)$ and $\mathsf{s}(t)$.
	\item[Step 2] According to the relation of $\mathsf{i}(t)$, $\mathsf{s}(t)$ and $\mathsf{s}(t+1)$ we choose an $\mathsf{i}(t)$ which would not make every two distinct $\mathsf{s}^{x}(t)$ , $\mathsf{s}^{y}(t)$$\in$ $\mathsf{S}(t)$ become the same state after affected by $\mathsf{i}(t)$ i.e. $\mathsf{s}^{x}(t+1)\ne\mathsf{s}^{y}(t+1)$. Such that, for every $\mathsf{s}^{x}(t+1)\in $ $\mathsf{S}(t+1)$ there is exact one corresponding $\mathsf{s}^{x}(t)\in $ $\mathsf{S}(t)$.
	\item[Step 3] Repeating Step 1 and Step 2 we can determine $\mathsf{s}(t)$, and for every $t>0$, for every $\mathsf{s}^{x}(t)\in \mathsf{S}(t)$ there is exact one corresponding $\mathsf{s}^{x}(t-1)\in \mathsf{S}(t-1)$, so we can determine $\mathsf{s}(t-1)$, \ldots, $\mathsf{s}(0)$.
\end{description}

%In the third observability, we can finish this procedure, thus we can determine the initial state \State$(0)$ for a \BCN. What is more, 


%Such that, the requirement for a \BCN\ to determine \State$(0)$ would be easist to satisfy. 
%{\color{red} (13)} 
Inspired by this procedure, we propose a new notion of observability named online observability to address the necessary and sufficient condition of determining $\mathsf{s}(0)$ by taking the determining procedure once. It states that a \BCN\ $\BB$ is observable if the previously mentioned procedure would terminates for every \State$(0)$. Its formal definitions will be presented in Section~\ref{sec:online}. 

We call this observability online observability, because we observe $\mathsf{o}(t)$ of \BCN\ and choose $\mathsf{i}(t)$ based on the information we have collected so far at every time $t$. 

In contrast, in the type-3 and type-4 observability we determine $\mathsf{s}(0)$ of \BCN\ by its recorded $\mathsf{o}(0)\mathsf{o}(1)\ldots\mathsf{o}(k)$ after we input $\mathsf{i}(0)\mathsf{i}(1)\ldots\mathsf{i}(k-1)$, and  we do not interfere with \BCN\ except for the logging of its inputs and outputs, thus we call them offline observability \cite{Cassar2017A}. %{\color{red} (13)}

%Then, in this paper we make the following contributions. 
\subsubsection*{Contributions}
Firstly, we propose and formally define the online observability. % to address the necessary and sufficient condition of determining $\mathsf{s}(0)$ by taking the determining procedure once. %It enriches the control-theory of \BCNs. 
Secondly, we also provide a determination algorithm for the online observability. Finally, we present some optimization brought by the online observability. Including the means to find shortest path and the approach to avoid entering critical states in the process of determining the initial states of \BCNs.  These optimization further explain the advantages of online observability of \BCNs. 

The remainder of this paper is organized as follows.

 {\em Section \ref{sec:pre}} introduces necessary preliminaries about \BCNs, including the definition of \BCNs\ and four existing observability. {\em Section \ref{sec:online}} presents the definition of online observability of \BCNs. {\em Section \ref{sec:deter}} presents the determination algorithm for online observability. 
% {\em Section \ref{sec:app}} talks about some optimization brought by the online observability of \BCNs. 
 {\em Section \ref{sec:con}} ends up with the introduction of our future work.

%==============================================================================================================