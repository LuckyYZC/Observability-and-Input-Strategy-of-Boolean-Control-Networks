% !Mode\dots ``TeX:UTF-8''
% !TEX root = ../bare_jrnl.tex
\section{Introduction}
\label{sec:intro}


\IEEEPARstart{I}n 1960s, Nobel Prize laureates Jacob and Monod~\cite{Jacob1961Genetic} found that ``Any cell contains a number of regulatory genes that act as switches and can turn one another on and off. If genes can turn one another on and off, then you can have genetic circuits.'' Inspired by these Boolean-type actions in genetic circuits, Boolean networks (\BNs) were firstly proposed by Kauffman \cite{Kauffman1968Metabolic} for modeling nonlinear and complex biological systems. 

{\BNs} are a type of discrete systems which are represented as directed graphs. In a Boolean network, each node has only two values ``0" and ``1", and
the nodes' values can change in different time steps.  For a node $n_i$, we use $n_i(t)$ to denote the value of $n_i$ at time step $t$.

The $n_i(t+1)$ is decided by a logical function of  set of  values  $\{n_j(t),\ldots,n_p(t)\}$ if  there are directed edges from $n_j,\ldots,n_p$ to $n_i$.  
 The logical operators used in  logical function including AND, OR, NO, XOR. 
Some general descriptions of the \BNs\ and their applications to biological systems can be found in~\cite{Kauffman1968Metabolic}.
There are many both natural and artificial systems works~\cite{Akutsu2000Inferring, Shmulevich2002From, Faur2006Dynamical,Green2007The,Lou2010Multi} related to \BNs.
 

\BNs\ can be naturally extended to Boolean control networks (\BCNs) when external regulations or perturbations are considered~\cite{Ideker2001A}. There are three kinds of nodes in \BCNs, including input-nodes ($\mathsf{i}$), state-nodes ($\mathsf{s}$) and output-nodes ($\mathsf{o}$). In \BCNs, we can only control input-nodes and observe output-nodes. 

For each state-node, $\mathsf{s}_i(t+1)$ is decided by a logical function of  $\{\mathsf{s}_j(t),\ldots,\mathsf{s}_p(t),\mathsf{i}_x(t),\ldots,\mathsf{i}_y(t)\}$  %if there are directed edges from $
while each $\mathsf{o}_i(t)$ is decided by a logical function of   $\{\mathsf{s}_j(t),\ldots,\mathsf{s}_p(t)\}$. 

\BCNs\ can be used to solve various real-life problems, for instance, 
structural and functional analysis of signalling and regulatory networks~\cite{Kaufman1999A, Klamt2006A}, 
abduction based drug target discovery~\cite{Biane2017Abduction}, 
and pursuing evasion problems in polygonal environments~\cite{Thunberg2011A}. 

We are mainly interested in the control-theoretic problems of \BCNs. The work~\cite{Akutsu2007Control} proves that the problem of determining the controllability of \BCNs\ is {\bf NP}-hard in the number of nodes. In addition, it points out that ``One of the major goals of systems biology is to develop a control theory for complex biological systems.'' Since then, the study on control-theoretic problems in the areas of \BNs\ and \BCNs\ has drawn great attention \cite{cheng2009controllability, Zhao2010Input, Cheng2011Identification, Cheng2011Analysis,Fornasini2013Observability}. What is more, the controllability and observability are the basic control-theoretic problems of \BCNs. % Among these studies, \emph{semi-tensor product} (\STP) is one of useful tools to deal with  

In this paper, we investigate the observability of \BCNs. The concept of observability was proposed firstly in~\cite{cheng2009controllability}. To date, there are four types of observability proposed in the literrature. And they are mainly about how to get  information of the initial value of the state-nodes of the \BCNs\ by the value of their input-nodes and output-nodes. 

For convenience, we use the vectors input $\mathsf{i}(t)=\begin{bmatrix}\mathsf{i}_1(t)\\ \vdots \\\mathsf{i}_m(t)\end{bmatrix}$, state $\mathsf{s}(t)=\begin{bmatrix}\mathsf{s}_1(t)\\ \vdots \\\mathsf{s}_n(t)\end{bmatrix}$ and output $\mathsf{o}(t)=\begin{bmatrix}\mathsf{o}_1(t)\\ \vdots \\\mathsf{o}_q(t)\end{bmatrix}$ to represent the value of all input-nodes, all state-nodes and all output-nodes, respectively, where $m$, $n$ and $q$ represent the number of input-nodes, state-nodes and output-nodes in the network, respectively. 
 Then we have $\mathsf{s}(t+1)$ is decided by $\mathsf{i}(t)$ and $\mathsf{s}(t)$, and $\mathsf{o}(t)$ is decided by $\mathsf{s}(t)$ as shown in Fig.~\ref{fig:10}.


 \begin{figure}[!t]
      \centering
      \framebox{\parbox{3in}{
		\centerline{\includegraphics[scale=0.28]{figures/Fig10.png}}
	}}
      
      \caption{The relationship of inputs, states and outputs.}
      \label{fig:10}
  \end{figure}
   The sequences $\mathsf{i}(0)\mathsf{i}(1)\ldots\mathsf{i}(k-1)$,  $\mathsf{s}(0)\mathsf{s}(1)\ldots\mathsf{s}(k)$, and $\mathsf{o}(0)\mathsf{o}(1)\ldots\mathsf{o}(k)$ 
 consists of several inputs, states and outputs in sequential time steps,  respectively, where $k>0$. 
 Such that, for an initial state $\mathsf{s}(0)$ and a  sequence of inputs $\mathsf{i}(0)\mathsf{i}(1)\ldots\mathsf{i}(k-1)$ of a \BCN, we have its corresponding 
$\mathsf{s}(0)\mathsf{s}(1)\ldots\mathsf{s}(k)$ and $\mathsf{o}(0)\mathsf{o}(1)\ldots\mathsf{o}(k)$.  
That is for a given  \BCN\  $\BB$,  $\mathsf{o}(0)\mathsf{o}(1)\ldots\mathsf{o}(k)$ is decide by
$\mathsf{s}(0)$ and the sequence $\mathsf{i}(0)\mathsf{i}(1)\ldots\mathsf{i}(k-1)$. 
 
We use $\mathsf{s}^{i}(t)$ and $\mathsf{s}^{j}(t)$ to represent different valuations of $\mathsf{s}(t)$, and similarly for input-nodes and output-nodes. Then the four existing observability of \BCNs\ can be described as follows. 

\begin{enumerate}
	\item The first type of observability was proposed in 2009 \cite{cheng2009controllability}, and it states that a \BCN\ $\BB$ is observable if for every $\mathsf{s}^{i}(0)$ there is an $\mathsf{i}(0)\mathsf{i}(1)\ldots\mathsf{i}(k-1)$ that can be used to distinguish $\mathsf{s}^{i}(0)$ from other types of initial states. That is in the $\BB$, for the $\mathsf{s}^{i}(0)$ and $\mathsf{i}(0)$$\mathsf{i}(1)\ldots$$\mathsf{i}(k-1)$, we have for every $\mathsf{s}^{j}(0)\ne\mathsf{s}^{i}(0)$, the corresponding $\mathsf{o}^{i}(0)$$\mathsf{o}^{i}(1)\ldots$$\mathsf{o}^{i}(k)$ of $\mathsf{s}^{i}(0)$ is different from the corresponding $\mathsf{o}^{j}(0)$$\mathsf{o}^{j}(1)\ldots$$\mathsf{o}^{j}(k)$ of $\mathsf{s}^{j}(0)$. 
	%------------------------------
	\item 
	The second observability was proposed in 2010 \cite{Zhao2010Input}, and it is determined in \cite{Li2015Controllability}. It states that a \BCN\ $\BB$ is observable if for every two distinct $\mathsf{s}^{i}(0)$ and $\mathsf{s}^{j}(0)$, there exists an $\mathsf{i}(0)$$\mathsf{i}(1)\ldots$$\mathsf{i}(k-1)$ that can be used to distinguish them. That is in the $\BB$, for the $\mathsf{s}^{i}(0)$, $\mathsf{s}^{j}(0)$ and $\mathsf{i}(0)\mathsf{i}(1)\ldots\mathsf{i}(k-1)$, we have the corresponding $\mathsf{o}^{i}(0)\mathsf{o}^{i}(1)\ldots\mathsf{o}^{i}(k)$ of $\mathsf{s}^{i}(0)$ and the corresponding $\mathsf{o}^{j}(0)\mathsf{o}^{j}(1)\ldots\mathsf{o}^{j}(k)$ of $\mathsf{s}^{j}(0)$ are different.
	\item The third observability proposed in 2011 \cite{Cheng2011Identification}, and it states that a \BCN\ $\BB$ is observable if there is an $\mathsf{i}(0)$$\mathsf{i}(1)\ldots$$\mathsf{i}(k-1)$ that determines $\mathsf{s}(0)$. That is in the $\BB$, for the $\mathsf{i}(0)$$\mathsf{i}(1)\ldots$$\mathsf{i}(k-1)$, we have for every two distinct $\mathsf{s}^{i}(0)$ and $\mathsf{s}^{j}(0)$, the corresponding $\mathsf{o}^{i}(0)$$\mathsf{o}^{i}(1)\ldots$$\mathsf{o}^{i}(k)$ of $\mathsf{s}^{i}(0)$ is different from the corresponding $\mathsf{o}^{j}(0)$$\mathsf{o}^{j}(1)\ldots$$\mathsf{o}^{j}(k)$ of $\mathsf{s}^{j}(0)$.
	
	\item  The fourth observability proposed in 2013 \cite{Fornasini2013Observability} is essentially the observability of linear control systems, i.e., that a \BCN\ $\BB$ is observable if every sufficient long $\mathsf{i}^{x}(0)$$\mathsf{i}^{x}(1)\ldots$$\mathsf{i}^{x}(k-1)$ can determine $\mathsf{s}(0)$. That is for every $\mathsf{i}^{x}(0)$$\mathsf{i}^{x}(1)\ldots$ $\mathsf{i}^{x}(k-1)$, for every two distinct $\mathsf{s}^{i}(0)$ and $\mathsf{s}^{j}(0)$, the corresponding $\mathsf{o}^{i}(0)$$\mathsf{o}^{i}(1)\ldots$$\mathsf{o}^{i}(k)$ of $\mathsf{s}^{i}(0)$ is different from the corresponding $\mathsf{o}^{j}(0)$$\mathsf{o}^{j}(1)\ldots$$\mathsf{o}^{j}(k)$ of $\mathsf{s}^{j}(0)$.
\end{enumerate}
 Their formal definitions will be presented in Section~\ref{sec:pre}.

 From the third and fourth notions of observability, we can see that one of the main problems of observability can been described as: 

\begin{problem}
\label{pro:1}
For a given \BCN\ $\BB$, whether or not there is a $\mathsf{i}(0)\mathsf{i}(1)\ldots\mathsf{i}(k-1)$, such that we can determine the initial state $\mathsf{s}(0)$ by this $\mathsf{i}(0)\mathsf{i}(1)\ldots\mathsf{i}(k-1)$?
\end{problem}

\begin{comment}
\ly{In the four existing observability, we can not determine the initial state of \BCNs\ by the first and second observability.} Although we can determine the initail state of \BCNs\ by the third and fourth observability, the requirements for \BCNs\ to determine the initail state are difficult to meet. Thus, we consider that whether we can determine the initial state of some \BCNs\ which can not be determined by the third and fourth observability.
\end{comment}
Although, these four existing notions of observability can help us study some information about $\mathsf{s}(0)$ we need. But when the determining procedure can do at most once, only third and fourth ones  can  determine the $\mathsf{s}(0)$ of \BCNs.
In other words, only third and fourth notions of observability can help us solve the {\em Problem~\ref{pro:1}}. What's more, in the third observability, if a \BCN\ is observable, then there has to exist an input sequence that determines its initial state $\mathsf{s}(0)$. Thus its condition is very strong, and the condition of fourth observability is even stronger.


However, we find that in order to determine $\mathsf{s}(0)$ we only need to complete the following procedure.

%Informly, the procedure of our  online observability as follows. 
For a given \BCN\  $\BB$. With its output sequence $\mathsf{o}(0)\mathsf{o}(1)\ldots\mathsf{o}(t)$ and input sequence $\mathsf{i}(0)\mathsf{i}(1)\ldots\mathsf{i}(t-1)$, we can infer the possible valuation of its $\mathsf{s}(t)$ by its updating rules. And  we use $\mathsf{S}(t)$ to denote the set of possible valuations of $\mathsf{s}(t)$. 
\begin{itemize}
	\item  Firstly, at every time step we infer $\mathsf{S}(t)$ by the $\mathsf{o}(t)$ we observed and the relation $\mathsf{o}(t)$ between $\mathsf{s}(t)$.
	\item Secondly, according to the relation of $\mathsf{i}(t)$, $\mathsf{s}(t)$ and $\mathsf{s}(t+1)$ we chose a $\mathsf{i}(t)$ which would make every two distinct $\mathsf{s}^{i}(t)$ , $\mathsf{s}^{j}(t)$$\in$ $\mathsf{S}(t)$ become the same state after affected by $\mathsf{i}(t)$ i.e. $\mathsf{s}^{i}(t+1)\ne\mathsf{s}^{j}(t+1)$. Therefore, for every $\mathsf{s}^{i}(t+1)\in $ $\mathsf{S}(t+1)$ there is exact one corresponding $\mathsf{s}^{i}(t)\in $ $\mathsf{S}(t)$.
	\item Thirdly, we have $|$$\mathsf{S}(t)$$|\le|$$\mathsf{S}(t-1)$$|$. When $|$$\mathsf{S}(t)$$|=1$, we can determine $\mathsf{s}(t)$. And then because for every $\mathsf{s}^{i}(t)\in $ $\mathsf{S}(t)$ there is exact one corresponding $\mathsf{s}^{i}(t-1)\in $ $\mathsf{S}(t-1)$ we can determine $\mathsf{s}(t-1)$, $\mathsf{s}(t-2)$, \ldots, and $\mathsf{s}(0)$.
\end{itemize}

\begin{comment}
 The input \Input$(t)$ we chose should make every two distinct states \State$^{i}(t)$ , \State$^{j}(t)$$\in$ \Ustate$(t)$ will not turn into be the same state after affected by \Input$(t)$.  \ly{Therefore,   there is exact one $s\in $ \Ustate$(t)$ such that $s\xrightarrow
{i(t)} s_1$ for each $s_1\in $ \Ustate$(t+1)$.} And the \Ustate$(t+1)$ is derived by the input \Input$(t)$ and \Output$(t+1)$, and we have $|$\Ustate$(t+1)$$|\le|$\Ustate$(t)$$|$. If $|$\Ustate$(t+1)$$|=1$,  we can determine \State$(t+1)$.  Employing the update rules and \Input$(t)$, we can determine the  \State$(t)$.  Repeating this step, we  determine the initial state \State$(0)$ of the \BCN.
\end{comment}

\begin{comment} 
 But we can also determine the set of possible initial states \Ustate$(0)$ by initial output \Output$(0)$ we observe, and then we can use different input sequences (\Input$^{1}(0)$\Input$^{1}(1)\ldots$\Input$^{1}(k)$, \Input$^{2}(0)$\Input$^{2}(1)\ldots$\Input$^{2}(k)$, $\ldots$) to determine initial state for different sets of possible initial states (\Ustate$^{1}(0)$, \Ustate$^{2}(0)$, $\ldots$). In this case, the requirements for \BCNs\ to determine the initail state would be easier to satisfy. 
\end{comment}

In the third observability, we can finish this procedure, so we can determine the initial state \State$(0)$ for a \BCN\ by third observability. What is more, from this proceture, we do not have to get the input sequence $\mathsf{i}(0)\mathsf{i}(1)\ldots\mathsf{i}(k-1)$ before we take the procedure of determining \State$(0)$. However, we can adaptively construct the input sequence $\mathsf{i}(0)\mathsf{i}(1)\ldots\mathsf{i}(k-1)$ to determine $\mathsf{s}(0)$ by the $\mathsf{S}(t)$ we derived at every time step. In this case, the requirement for a \BCN\ to determint \State$(0)$ would be easier to satisfy. Inspired by this, we propose the online observability in this paper which states that a \BCN\ $\BB$ is observable if the $\mathsf{i}(0)\mathsf{i}(1)\ldots\mathsf{i}(k-1)$ to determine $\mathsf{s}(0)$ can be adaptively constructed. 

Comparing with the existing first and second observability, the online observability can determine the initial states of \BCNs. However, comparing with the existing third and fourth observability, in online observability, the requirements for \BCNs\ to determine the initail state is easier to satisfy.  
\begin{comment} 
In order to study online observability better, we need to formulate the formal definition for it. Firstly, we define the derivation function to describe the derivation process of the state of \BCNs\ by their output and input at every time step. Secondly, we propose the definition of the $k$-step determinability to present that we can use  \Ustate$(t)$ to determine the state of \BCNs\ \State$(t)$ in $k$ time steps. The derivation function and $k$-step determinability are the preparations for defining the the online observability. 
Thirdly, we define the online observability that for every \Ustate$(0)$, there exists an $k^i$ such that \Ustate$(0)$ is $k^i$-step determinable. By the definition, we prove that online observability is the necessary and sufficient condition of determine the initial state of \BCNs\ for every initial state. Finally, we compare the online observability with existing four observability. That the online observability implies the first and second observability but does not imply the third and fourth observability.

After we defined the online observability and campred it with with existing four observability, we propose two algorithms to determine it. The first one is the supertree-based algorithm, the supertree of a \BCN\ intuitively depicts how to derive the state of the \BCN\ by alternately observing the output and deciding the input untill we can determine the state of \BCN\ \State$(t)$. But there are some shortcomings in this algorithm, thus we propose the algorithm based on directed graph. In the algorithm based on directed graph, we check whether the $k$-step determinability of the sets with fewer possible states and then check the sets with more possible states. So that, it can help us find all paths to determine the initial state of \BCNs.

Finally, we further illustrate the advantages of the online observability. Because the online observability has better performance than existing observability in the dynamic analysis of \BCNs.
We can use it to do some optimization in the process of determining the initial state. The first one is to find the shortest path and the second one is to avoid entering critical states. 
\end{comment}
Then, in this paper we make the following contributions. 

\subsubsection*{Contributions}
Firstly, we propose and formally define the concept of online observability of \BCNs. Comparing with existing observability, the online observability can help to determine the initial states of some \BCNs\ which can not be determined before. Secondly, in addition to theoretical research, we also provide two algorithms to determine the online observability for \BCNs. Finally, we present some optimization brought by the online observability. Including the means to find shortest path and the approach to avoid entering critical states in the process of determining the initial states of \BCNs.  These optimization further explain the advantages of online observability of \BCNs. 

The remainder of this paper is organized as follows.

 {\em Section \ref{sec:pre}} introduces necessary preliminaries about \BCNs, including the definition of \BCNs\ and four existing observability of \BCNs. {\em Section \ref{sec:online}} presents the definition of online observability of \BCNs. {\em Section \ref{sec:deter}} presents the algorithms to determine the online observability of \BCNs. {\em Section \ref{sec:app}} talks about some optimization brought by the online observability of \BCNs. {\em Section \ref{sec:con}} ends up with the introduction of our future work.

%==============================================================================================================