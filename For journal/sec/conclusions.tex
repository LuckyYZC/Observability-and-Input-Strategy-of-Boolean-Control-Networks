%<<<<<<< HEAD
%<<<<<<< HEAD
%<<<<<<< HEAD
% !Mode\dots ``TeX:UTF-8''
% !TEX root = ../root.tex
\section{Conclusions}
\label{sec:con}
In this paper, firstly we proposed the online observability of \BCNs\ and define its mathematical form. Secondly we propose the algorithm based on the input-labelled graph to determine the online observability. %After that we present some optimization brought by the online observability to further illustrate its advantages. 


But even with the algorithm based on input-labelled graph, it is still hard to determine the online observability of large scale \BCNs. Therefore, in the future we will try to separate the \BCN\ into several subnets to determine their online observability respectively, and then determine the online observability of whole \BCN. Furthermore, we also want to try to use some knowledge about formal methods to earn scalability for \BCNs. 

%In addition to the theoretical aspect, the realistic application is also very important. Thus we will also try to find some realistic examples which can be modeled by \BCNs. So that we can use the online observability to analyze them to gain better performance.