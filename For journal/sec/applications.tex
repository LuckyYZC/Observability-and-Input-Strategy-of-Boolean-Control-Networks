% !Mode\dots ``TeX:UTF-8''
% !TEX root = ../root.tex
\section{Optimization}
\label{sec:app}

In the {\em Section \ref{sec:online}}, we present that a  \BCN\ is online observable iff its initial state $\mathsf{s}(0)$ can be determined. Therefore, online observability can help us determine the initial state of some \BCNs\ which can not be determined by the existing third and fourth observability. In addition, we can also use the online observability to some optimization in the process of determining the initial state of the \BCNs, including finding the shortest path and avoid entering critical states. %Because the output we observe is not sure, we use expected value and variance of the length of path and the times of entering critical states to help us to choose the input.If we research the systems described by \BCNs\ which are online observable but not satisfy the existing third and fourth observability, we can only use the online observability to determine their initial state in real time. If we have built the corresponding directed graphs for them, we can use the online observability to determine the initial state of these \BCNs. In addition, online observibility requires less observation costs for us to determine the initial state of some systems described by  \BCNs\ in real time. 


\subsection{Finding shortest path}
When we need to determine the initial state of a \BCN, an important aspect that we would consider is to find the shortest path to determine the initial state. In general, we can not find the shortest path definitely because the output is uncertain. Fortunately, we can use the directed graph to make the best decision. For the path to determine the initial state of the \BCNs, we introduce two functions $\Pe(\mathsf{S},\mathsf{ i}^{i})$ and $\Pv(\mathsf{S},\mathsf{ i}^{i})$ to describe the expected value and variance for every node $\mathsf{S}$ of the directed graph, respectively. Such that, at every time step, with $\mathsf{S}(t)$ we derived, we can chose $\mathsf{i}(t)$ by $\Pe(\mathsf{S}(t),\mathsf{ i}(t))$ and $\Pv(\mathsf{S},\mathsf{ i}^{i})$ to try to find the shortest path. 
%We introduce two functions $Spe(S_i, I_i)$ and $Spv(S_i, I_i)$ to describe the expected value of the shortest path and variance of shortest path , $S_i$ is the possible states set and $I_i$ is the input we chose, the definition of $Spe(S_i, I_i)$ is as follows:\\
%With the $Spe(S_i, I_i)$ and $Spv(S_i, I_i)$ we can make the decision we like.


We introduce some necessary statements before defining the functions $\Pe(\mathsf{S},\mathsf{ i}^{i})$ and $\Pv(\mathsf{S},\mathsf{ i}^{i})$:
\begin{itemize}
  \item $\mathsf{S}$: the set of states of a node in the directed graph;
  \item $\{\mathsf{ i}^{1},\mathsf{ i}^{2},\ldots, \mathsf{ i}^{p}\}$ : the inputs set of $\mathsf{S}$;
  \item $\{\mathsf{o}^1,\mathsf{o}^2,\ldots,\mathsf{o}^k\}$: the set of possible outputs of \BCN\ at next time step as we input $\mathsf{ i}^{i}$;
 \item $\{\mathsf{S}^{1},\mathsf{S}^{2},\ldots, \mathsf{S}^{k}\}$ : the set of possible state sets at next time step that for every $\mathsf{S}^{j}$ \[\mathsf{S}^{j}=\Ded\left(\mathsf{S},\mathsf{i}^{i},\mathsf{o}^{j}\right).\] 
  
 % \item the fourth one implies the third one, second one and first one.
\end{itemize} 

Then the definition of $\Spe(\mathsf{S})$ and $\Pe(\mathsf{S},\mathsf{ i}^{i})$ are as follows.

\begin{definition}[$\Spe(\mathsf{S})$] \label{lspe}
\begin{equation}
\begin{split}
\Spe(\mathsf{S})= \min(\Pe(\mathsf{S}, \mathsf{ i}^{1}),\ldots,\Pe(\mathsf{S}, \mathsf{ i}^{p})).
\end{split}
\end{equation}
 %\[\Spe(\mathsf{S}(t))= \min(\Pe(\mathsf{S}(t), \mathsf{ i}^{1}(t)),\ldots,\Pe(\mathsf{S}(t), \mathsf{ i}^{z}(t))).\] 
\end{definition}

\begin{definition}[$\Pe(\mathsf{S}, \mathsf{ i}^{i})$] \label{spe}

When the $|\mathsf{S}|=1$,
\[\Pe(\mathsf{S}, \mathsf{ i}^{i})=0\]  for every \[\mathsf{ i}^{i} \in \{\mathsf{ i}^{1},\mathsf{ i}^{2},\ldots, \mathsf{ i}^{p}\}.\] According to {\em Definition \ref{lspe}}, $\Spe(\mathsf{S})=0$ if $|\mathsf{S}|=1$. 

When the $|\mathsf{S}|>1$, 
%and the $\{I_1,I_2,\cdots, I_p\}$ is the right inputs set of $S_i$. For every $I_i$ in $\{I_1,I_2,\cdots, I_p\}$ the $\{S_i^1,S_i^2,\cdots, S_i^k\}$ is a set of state sets whose elements corresponding to the possible outputs after input $I_i$, then 
we have that  
\begin{equation}
\begin{split}
\Pe(\mathsf{S}, \mathsf{ i}^{i})=1 +\frac{\sum_{j=1}^k \Spe(\mathsf{S}^{j})|\mathsf{S}^{j}|}{ |\mathsf{S}|}
\end{split}
\end{equation}
%\[\Pe(\mathsf{S}(t), \mathsf{ i}^{i}(t))=1 +\frac{\sum_{j=1}^k \Spe(\mathsf{S}^{j}(t+1))|\mathsf{S}^{j}(t+1)|}{ |\mathsf{S}(t)|}\] 
%and 
%\[{\tt Spe}(S_i)= \min({\tt Pe}(S_i, I_1),{\tt Pe}(S_i, I_2),\cdots,{\tt Pe}(S_i, I_p))\]
\end{definition}

The function $\Spe(\mathsf{S})$ is used to find the $\mathsf{ i}^{i}$ from $\{\mathsf{ i}^{1},\mathsf{ i}^{2},\ldots, \mathsf{ i}^{z}\}$ to calculat least $\Pe(\mathsf{S}, \mathsf{ i}^{i})$ for $\mathsf{S}$. In the definition of $\Pe(\mathsf{S}, \mathsf{ i}^{i})$, if $|\mathsf{S}|=1$ then we can determine $\mathsf{s}$. Thus we no longer need to choose the input to determine the state $\mathsf{s}$ of \BCNs. Therefore, for any input the path expected value $\Pe(\mathsf{S}, \mathsf{ i}^{i})$ would be $0$ and the shortest path expected value $\Spe(\mathsf{S})$ also would be $0$ too. While, if $|\mathsf{S}|>1$, then we still need to choose input and observe the output to determine the state of of \BCNs. Thus we recursively define the $\Pe(\mathsf{S}, \mathsf{ i}^{i})$ and $\Spe(\mathsf{S})$ for $\mathsf{S}$ and each input $\mathsf{ i}^{i}$ in the inputs set of $\mathsf{S}$. 

If we want to find the shortest path to determine the initial state of a \BCN, we can choose an input $\mathsf{ i}^{i}$ with least $\Pe(\mathsf{S}, \mathsf{ i}^{i})$ by the function $\Spe(\mathsf{S})$. This input $\mathsf{ i}^{i}$ may help us find the shortes path to determine the state. But the output of \BCNs\ we observe is uncertain after we choose the input $\mathsf{ i}^{i}$, hence selecting the $\mathsf{ i}^{i}$ which with least $\Pe(\mathsf{S}, \mathsf{ i}^{i})$ may leads to a very long path to determine the initial state of \BCNs. For better performce, we define the $\Pv(\mathsf{S}, \mathsf{ i}^{i})$ to avoid risks.% The $\Pv(S, i_p)$ is defined in the similar way. %Hence we omit the details of the definition of $\Pv(S, i_p)$ in this paper. \\
\begin{definition}[$\Pv(\mathsf{S}, \mathsf{ i}^{i})$] 
When the $|\mathsf{S}|=1$, we have that
\[\Pv(\mathsf{S}, \mathsf{ i}^{i})=0\]  for every \[\mathsf{ i}^{i} \in \{\mathsf{ i}^{1},\mathsf{ i}^{2},\ldots, \mathsf{ i}^{z}\}.\]  When the $|\mathsf{S}|>1$, 
%and the $\{I_1,I_2,\cdots, I_p\}$ is the right inputs set of $S_i$. For every $I_i$ in $\{I_1,I_2,\cdots, I_p\}$ the $\{S_i^1,S_i^2,\cdots, S_i^k\}$ is a set of state sets whose elements corresponding to the possible outputs after input $I_i$, then 
we have that\\ 
\begin{equation}
\begin{split}
\Pv(\mathsf{S}, \mathsf{ i}^{i})=\frac{\sum_{j=1}^k (\Spe(\mathsf{S}^{j})-\Pe(\mathsf{S}, \mathsf{ i}^{i})+1)^2 |\mathsf{S}^{j}|}{ |\mathsf{S}|}
\end{split}
\end{equation}%$\Pv(\mathsf{S}(t), \mathsf{ i}^{i}(t))=
%\frac{\sum_{j=1}^k (\Spe(\mathsf{S}^{j}(t+1))-\Pe(\mathsf{S}(t), \mathsf{ i}^{i}(t))+1)^2 |\mathsf{S}^{j}(t+1)|}{ |\mathsf{S}(t)|}$
%and 
%\[{\tt Spe}(S_i)= \min({\tt Pe}(S_i, I_1),{\tt Pe}(S_i, I_2),\cdots,{\tt Pe}(S_i, I_p))\]
\end{definition}

From the definition of the $\Pv(\mathsf{S}, \mathsf{ i}^{i})$, we have that if the $\Pv(\mathsf{S}, \mathsf{ i}^{i})$ of input $\mathsf{ i}^{i}$ is not very large, the risk of choosing $\mathsf{ i}^{i}$ would be not very great either.
\subsection{Avoiding entering critical states}
In biological systems wich depiected by the \BCNs, some of the genes' states may corresponding to unfavorable or even dangerous situations \cite{Li2014Controllability}. So another important aspect that we consider is to avoid entering critical states in the process of determining the \BCN's initial state. Therefore, we also construct two functions $\Ce(\mathsf{S}, \mathsf{ i}^{i})$ and $\Cv(\mathsf{S}, \mathsf{ i}^{i})$ to describe expected value and variance of the times of entering critical states in the process determine the intial state for every node $\mathsf{S}$ of the directed graph, respectively. The definition of $\Lce(\mathsf{S})$ and $\Ce(\mathsf{S}, \mathsf{ i}^{i})$ are as follows.\\
\begin{definition}[$\Lce(\mathsf{S})$] \label{lce}
\begin{equation}
\begin{split}
\Lce(\mathsf{S})= \min(\Ce(\mathsf{S}, \mathsf{ i}^{1}),\ldots,\Ce(\mathsf{S}, \mathsf{ i}^{p})).
\end{split}
\end{equation}
%\[\Lce(\mathsf{S}(t))= \min(\Ce(\mathsf{S}(t), \mathsf{ i}^{1}(t)),\ldots,\Ce(\mathsf{S}(t), \mathsf{ i}^{z}(t))).\]
\end{definition}
\begin{definition}[$\Ce(\mathsf{S}, \mathsf{ i}^{i})$] 
When the $|\mathsf{S}|=1$, we have \[\Ce(\mathsf{S}, \mathsf{ i}^{i})=|\mathsf{S} \cap \mathsf{S}_{cri} |,\]  for every \[\mathsf{ i}^{i} \in \{\mathsf{ i}^{1},\mathsf{ i}^{2},\ldots, \mathsf{ i}^{p}\}.\]  
According {\em Definition \ref{lce}}, %{\tt Spe}$(S_i)=0$ if $|S_i|=1$ 
\[\Lce(\mathsf{S})=\Ce(\mathsf{S}, \mathsf{ i}^{i})=|\mathsf{S} \cap \mathsf{S}_{cri} |\] 
But when the $|\mathsf{S}|>1$ 
%and the $\{I_1,I_2,\cdots, I_p\}$ is the right inputs set of $S_i$. For every $ I_i$ in $\{I_1,I_2,\cdots, I_p\}$ the $\{S_i^1,S_i^2,\cdots, S_i^k\}$ is a set of state sets whose elements corresponding to the possible outputs after input $I_i$, then 
we have that 
\begin{equation}
\begin{split}
\Ce(\mathsf{S}, \mathsf{ i}^{i})=|\mathsf{S} \cap \mathsf{S}_{cri}| +\frac{\sum_{j=1}^k \Lce(\mathsf{S}^{j})|\mathsf{S}^{j}|}{ |\mathsf{S}|} 
\end{split}
\end{equation}
%\[\Ce(\mathsf{S}(t), \mathsf{ i}^{i}(t))=|\mathsf{S}(t) \cap \mathsf{S}_{cri}| +\frac{\sum_{j=1}^k \Lce(\mathsf{S}^{j}(t+1))|\mathsf{S}^{j}(t+1)|}{ |\mathsf{S}(t)|} \] 
%and 
%\[{\tt Lce}(S_i)= \min({\tt Ce}(S_i, I_1),{\tt Ce}(S_i, I_2),\cdots,){\tt Ce}(S_i, I_p)\]
\end{definition}

Where $\mathsf{S}_{cri}$ is the set of critical states of the \BCN\ we research. The definition of $\Ce(\mathsf{S}, \mathsf{ i}^{i})$ has some difference with $\Pe(\mathsf{S}, \mathsf{ i}^{i})$, because of the critical states set $\mathsf{S}_{cri}$. So that we can analyze the possibility of entering the critical states by caculate $\Ce(\mathsf{S}, \mathsf{ i}^{i})$ for every node in the directed graph, and we can also get the definitions of $\Cv(\mathsf{S}, \mathsf{ i}^{i})$ in the similar way. %We omit the details of the definition of $\Cv(S, i_p)$ in this paper as well. 

\begin{definition}[$\Cv(\mathsf{S}, \mathsf{ i}^{i})$] 
When the $|\mathsf{S}|=1$, we have that
$\Cv(\mathsf{S}, \mathsf{ i}^{i})=0$ for every  \[\mathsf{ i}^{i} \in \{\mathsf{ i}^{1},\mathsf{ i}^{2},\ldots, \mathsf{ i}^{z}\}.\]   When the $|\mathsf{S}|>1$, 
%and the $\{I_1,I_2,\cdots, I_p\}$ is the right inputs set of $S_i$. For every $I_i$ in $\{I_1,I_2,\cdots, I_p\}$ the $\{S_i^1,S_i^2,\cdots, S_i^k\}$ is a set of state sets whose elements corresponding to the possible outputs after input $I_i$, then 
we have that  \\
\begin{equation}
\begin{split}
\Cv(\mathsf{S}, \mathsf{ i}^{i})=\sum_{j=1}^k\frac{ (\Lce(\mathsf{S}^{j})-\Ce(\mathsf{S}, \mathsf{ i}^{i})+|\mathsf{S} \cap \mathsf{S}_{cri}|)^2 |\mathsf{S}^{j}|}{ |\mathsf{S}|}
\end{split}
\end{equation}
%$\Cv(\mathsf{S}(t), \mathsf{ i}^{i}(t))=
      %     \frac{\sum_{j=1}^k (\Lce(\mathsf{S}^{j}(t+1))-\Ce(\mathsf{S}(t), \mathsf{ i}^{i}(t))+|\mathsf{S}(t) \cap \mathsf{S}_{cri}|)^2 |\mathsf{S}^{j}(t+1)|}{ |\mathsf{S}(t)|}$
%and 
%\[{\tt Spe}(S_i)= \min({\tt Pe}(S_i, I_1),{\tt Pe}(S_i, I_2),\cdots,{\tt Pe}(S_i, I_p))\]
\end{definition}

The use of $\Ce(\mathsf{S}, \mathsf{ i}^{i})$ and $\Cv(\mathsf{S}, \mathsf{ i}^{i})$ are similar to $\Pe(\mathsf{S}, \mathsf{ i}^{i})$ and $\Pv(\mathsf{S}, \mathsf{ i}^{i})$ respectively. They help us avoid entering critical states of \BCNs\ in the process of determining the initial state of \BCNs. Therefore, at every time step, with $\mathsf{S}(t)$ we derived, we can chose $\mathsf{i}(t)$ by $ \Pe(\mathsf{S}(t), \mathsf{ i}(t))$, $\Pv(\mathsf{S}(t), \mathsf{ i}(t))$, $\Ce(\mathsf{S}(t), \mathsf{ i}(t))$ and $\Cv(\mathsf{S}(t), \mathsf{ i}(t))$ to get better performance.

However, in the four existing types of observability, we can not analyze the state of the \BCNs\ dynamically. Hence it would be hard to do some optimization in the process of determining the initial state of the \BCNs. Fortunately, this problem can be solved better by the online observability of the \BCNs, thus it further illustrates the advantages of online observability.