\begin{appendices}
\section{Proof}
\label{sec:pro}
\subsection{Proof of Lemma \ref{lemm:1}}
\begin{proof}
%$\mathsf{S}^{1}\subseteq \mathsf{S}^{2}$, 
If $\Ks(\mathsf{S}^{2})=0$, we have  \[\mathsf{S}^{1} = \mathsf{S}^{2}.\] Therefore, we have $\Ks(\mathsf{S}^{1})=0\le\Ks(\mathsf{S}^{2})$, thus we have the {\em Lemma \ref{lemm:1}} is right when $\Ks(\mathsf{S}^{2})=0$.
 
 If for $\Ks(\mathsf{S}^{2})=0,\ldots, p$ the {\em Lemma \ref{lemm:1}} is right. When $\Ks(\mathsf{S}^{2})=p+1$, we have for $\mathsf{S}^{2}$ there exists a $\mathsf{i}^{x}\in \Delta_M$ such that
 \begin{itemize}
 \item  $|\Ded\left(\mathsf{S}^{2},\mathsf{i}^{x},\varepsilon\right)|=|\mathsf{S}^{2}|$, and 
 \item  $\Ks(\Ded(\mathsf{S}^{2},\mathsf{i}^{x},\mathsf{o}^{j}))<(p+1)$, for each non-empty $\Ded(\mathsf{S}^{2},\mathsf{i}^{x},\mathsf{o}^{j})$.
 \end{itemize}
 and there is not any $\mathsf{i}^{y} \in \Delta_M$ such that
  \begin{itemize}
 \item  $|\Ded\left(\mathsf{S}^{2},\mathsf{i}^{y},\varepsilon\right)|=|\mathsf{S}^2|$,
 \item  $\Ks(\Ded\left(\mathsf{S}^{2},\mathsf{i}^{y},\mathsf{o}^{j}\right))< p$ for each non-empty $\Ded\left(\mathsf{S}^{2},\mathsf{i}^{y},\mathsf{o}^{j}\right)$.
 \end{itemize} 
 Then we have for $\mathsf{S}^{1}$ there exists a $\mathsf{i}^{z}\in \Delta_M$ such that
 \begin{itemize}
 \item  $|\Ded\left(\mathsf{S}^{1},\mathsf{i}^{z},\varepsilon\right)|=|\mathsf{S}^{1}|$, and 
 \item  $\Ks(\Ded(\mathsf{S}^{1},\mathsf{i}^{z},\mathsf{o}^{j}))<p'\le p+1$, for each non-empty $\Ded(\mathsf{S}^{1},\mathsf{i}^{z},\mathsf{o}^{j})\subseteq \Ded(\mathsf{S}^{2},\mathsf{i}^{z},\mathsf{o}^{j})$.
 \end{itemize}
 and there is not any $\mathsf{i}^{w} \in \Delta_M$ such that
  \begin{itemize}
 \item  $|\Ded\left(\mathsf{S}^{1},\mathsf{i}^{y},\varepsilon\right)|=|\mathsf{S}^1|$,
 \item  $\Ks(\Ded\left(\mathsf{S}^{1},\mathsf{i}^{y},\mathsf{o}^{j}\right))<p'\le p$ for each non-empty $\Ded\left(\mathsf{S}^{1},\mathsf{i}^{w},\mathsf{o}^{j}\right)\subseteq \Ded\left(\mathsf{S}^{2},\mathsf{i}^{w},\mathsf{o}^{j}\right)$,
 \end{itemize} 
 Therefore we have if for $\Ks(\mathsf{S}^{2})=0,\ldots, p$ the {\em Lemma \ref{lemm:1}} is right, then the {\em Lemma \ref{lemm:1}} is right too when $\Ks(\mathsf{S}^{2})=p+1$. 
And we have the {\em Lemma \ref{lemm:1}} is right when $\Ks(\mathsf{S}^{2})=0$. Thus we have the {\em Lemma \ref{lemm:1}} is right for every $\Ks(\mathsf{S}^{2})\ne \infty$.
 
\end{proof}
\subsection{Proof of Lemma \ref{lemm:2}}
\begin{proof}
For the non-empty $\mathsf{S}^{1}$ and $\mathsf{S}^{2}$, if $\mathsf{S}^{1}\subseteq\mathsf{S}^{2}$ and $\Ks(\mathsf{S}^{2})\ne\infty$.
For every $\mathsf{i}^1\in \Ri(\mathsf{S}^2)$, we have  $|\Ded\left(\mathsf{S}^2,\mathsf{i}^1,\varepsilon\right)|=|\mathsf{S}^2|$, and 
for every $\mathsf{o} \in \Delta_Q$, $\Ded(\mathsf{S}^2,\mathsf{i}^1,\mathsf{o})\ne \emptyset$ implies $\Ks(\Ded(\mathsf{S}^2,\mathsf{i}^1,\mathsf{o}))\ne \infty$. With {\em Lemma \ref{lemm:1}} we have $|\Ded\left(\mathsf{S}^1,\mathsf{i}^1,\varepsilon\right)|=|\mathsf{S}^1|$, and 
for every $\mathsf{o} \in \Delta_Q$, $\Ded(\mathsf{S}^1,\mathsf{i}^1,\mathsf{o})\ne \emptyset$ implies $\Ks(\Ded(\mathsf{S}^1,\mathsf{i}^1,\mathsf{o}))\le \Ks(\Ded(\mathsf{S}^2,\mathsf{i}^1,\mathsf{o}))$, thus $\mathsf{i}^1\in \Ri(\mathsf{S}^1)$. Therefore, $\Ri(\mathsf{S}^2)\subseteq\Ri(\mathsf{S}^1)$.
\end{proof}


\subsection{Proof of Lemma \ref{lemm:3}}
\begin{proof} Firsrly, we prove the propostion that for a set of possible state $\mathsf{S}(t)$ if there exists a $k^{1}\ge 0$ that $\mathsf{S}(t)$ is $k^{1}$-step determinable, then for every state \State$^{x}(t)\in \mathsf{S}(t)$, there exists an input sequence $\mathsf{I}\in(\Delta_M)^{k^2}$ for some $k^2 >0$ such that for every $\mathsf{s}^{y}(t)\in \mathsf{S}(t)$, $\mathsf{s}^{y}(t)\neq \mathsf{s}^{x}(t)$ implies $(HF)^{k^2}_{\mathsf{s}^{y}(t)}(\mathsf{I})\neq (HF)^{k^2}_{{\mathsf{s}^{x}(t)}}(\mathsf{I})$.
\begin{itemize}
\item When $k^{1}=0$, we have $|\mathsf{S}(t)|=1$, then for every \State$^{x}(t)$$\in \mathsf{S}(t)$ there does not exist any $\mathsf{s}^{y}(t)\in \mathsf{S}(t)$ and $\mathsf{s}^{y}(t)\neq \mathsf{s}^{x}(t)$. Therefore, we have that for any $k^2 >0$, for every input sequence $\mathsf{I}\in(\Delta_M)^{k^2}$ the propostion is right. 
\item If for $k^{1}=0,\ldots, p$ the propostion is right. When $k^{1}=p+1$, we have for $\mathsf{S}(t)$ there exists a $\mathsf{i}^{i}(t)\in \Delta_M$ such that
 \begin{itemize}
 \item  $|\Ded\left(\mathsf{S}(t),\mathsf{i}^{i}(t),\varepsilon\right)|=|\mathsf{S}(t)|$, and 
 \item  for each \[\mathsf{o}^{j}(t+1)\in \Delta_Q\] such that \[\Ded(\mathsf{S}(t),\mathsf{i}^{i}(t),\mathsf{o}^{j}(t+1))\neq \emptyset\] and $\Ded(\mathsf{S}(t),\mathsf{i}^{i}(t),\mathsf{o}^{j}(t+1))$ is $k'$-step determinable where ${k'}<(p+1)$.
 \end{itemize}
 Then we have for every \State$^{x}(t)$$\in \mathsf{S}(t)$, there exists an $\mathsf{I}\in(\Delta_M)^{k^2}$ for some $k^2 >0$, such that for all $\mathsf{s}^{y}(t)\in \mathsf{S}(t)$, $\mathsf{s}^{y}(t)\neq \mathsf{s}^{x}(t)$ implies $(HF)^{k^2}_{\mathsf{s}^{y}(t)}(\mathsf{I})\neq (HF)^{k^2}_{{\mathsf{s}^{x}(t)}}(\mathsf{I})$, where the input sequence $\mathsf{I}=\mathsf{i}^{i}(t)\mathsf{I}'$. That the input sequence $\mathsf{I}'$ satisfies the following properties,
  for all states \[\mathsf{s}^{z}(t+1)\in \Ded(\mathsf{S}(t),\mathsf{i}^{i}(t),h(\mathsf{s}^{x}(t+1))),\]\[\mathsf{s}^{z}(t+1)\neq \mathsf{s}^{x}(t+1)\] implies \[(HF)^p_{\mathsf{s}^{z}(t+1)}(\mathsf{I}')\neq (HF)^p_{{\mathsf{s}^{x}(t+1)}}(\mathsf{I}'),\] where $\mathsf{s}^{x}(t+1)=f(\mathsf{s}^{x}(t),\mathsf{i}^{i}(t))$. Then we have the propostion is right when $k^{1}=p+1$. 

\end{itemize}
As the propostion is right when $k^{1}=0$, and we have if for $k^{1}=0,\ldots, p$ the propostion is right, then the propostion is right when $k^{1}=p+1$. Thus the propostion is right for any $k^{1}\ge0$.

Secondly, we have that if a \BCN\ is online observable,
then for every  \[\mathsf{o}^{j}(0)\in \Delta_Q\] such that \[\Ded\left(\Delta_N,\varepsilon, \mathsf{o}^{j}(0)\right)\ne \emptyset,\] there exists a $k^{j}\ge0$ that $\Ded\left(\Delta_N,\varepsilon,\mathsf{o}^{j}(0)\right)$ is $k^{j}$-step determinable. 

Therefore, we have for every initial state \State$^{x}(0)$$\in \Delta_N$, there exists an input sequence $\mathsf{I}\in(\Delta_M)^{k^i}$ for some $k^i >0$ such that for all states $\mathsf{s}^{y}(0)\neq \mathsf{s}^{x}(0)$, $h(\mathsf{s}^{y}(0))=h(\mathsf{s}^{x}(0))$ implies $(HF)^{k^i}_{\mathsf{s}^{y}(0)}(\mathsf{I})\neq (HF)^{k^i}_{{\mathsf{s}^{x}(0)}}(\mathsf{I})$. Thus the \BCN\ satisfies the first observability if it is online observable.
\end{proof}

\subsection{Proof of Lemma \ref{lemm:4}}

\begin{proof}
Firsrly, we prove the propostion that for a set of possible state $\mathsf{S}(t)$, if there exists an input sequence $\mathsf{I}\in(\Delta_M)^{k^1}$ for some $k^1 >0$, such that for any distinct states $\mathsf{s}^{x}(t)$, $\mathsf{s}^{y}(t) \in \mathsf{S}(t)$, implies $(HF)^{k^1}_{\mathsf{s}^{x}(t)}(\mathsf{I})\neq (HF)^{k^1}_{\mathsf{s}^{y}(t)}(\mathsf{I})$, then for the $\mathsf{S}(t)$ there exists a $k^{2}\ge 0$ such that $\mathsf{S}(t)$ is $k^{2}$-step determinable.

\begin{itemize}
\item When $k^1=1$, for any distinct states $\mathsf{s}^{x}(t)$, $\mathsf{s}^{y}(t) \in \mathsf{S}(t)$, implies $(HF)^{k^1}_{\mathsf{s}^{x}(t)}(\mathsf{I})\neq (HF)^{k^1}_{\mathsf{s}^{y}(t)}(\mathsf{I})$. Then we have for $\mathsf{S}(t)$,
 there exists an input $\mathsf{i}^{i}(t)=\mathsf{I}$ such that
 \begin{itemize}
 \item  $|\Ded\left(\mathsf{S}(t),\mathsf{i}^{i}(t),\varepsilon\right)|=|\mathsf{S}(t)|$, and 
 \item  for each \[\mathsf{o}^{j}(t+1)\in \Delta_Q\] such that \[|\Ded\left(\mathsf{S}(t),\mathsf{i}^{i}(t),\mathsf{o}^{j}(t+1)\right)|=1,\] the $\Ded\left(\mathsf{S}(t),\mathsf{i}^{i}(t),\mathsf{o}^{z}(t+1)\right)$ is $0$-step determinable.
 \end{itemize}
Thus the $\mathsf{S}(t)$ is $1$-step determinable, then the propostion is right when $k^1 =1$.

\item If for $k^1=1,\ldots, p$ the propostion is right. Then when $k^1=p+1$, for any distinct states $\mathsf{s}^{x}(t)$, $\mathsf{s}^{y}(t) \in \mathsf{S}(t)$, implies $(HF)^{k^1}_{\mathsf{s}^{x}(t)}(\mathsf{I})\neq (HF)^{k^1}_{\mathsf{s}^{y}(t)}(\mathsf{I})$. Then we have for $\mathsf{S}(t)$,
 there exists an input $\mathsf{i}^{i}(t)$ which is the first input of $\mathsf{I}$, such that
 \begin{itemize}
\item  $|\Ded\left(\mathsf{S}(t),\mathsf{i}^{i}(t),\varepsilon\right)|=|\mathsf{S}(t)|$, and 
 \item  for each \[\mathsf{o}^{j}(t+1)\in \Delta_Q\] such that \[\Ded\left(\mathsf{S}(t),\mathsf{i}^{i}(t),\mathsf{o}^{j}(t+1)\right)\ne \emptyset,\]  the $\Ded\left(\mathsf{S}(t),\mathsf{i}^{i}(t),\mathsf{o}^{j}(t+1)\right)$ is $p$-step determinable.
 \end{itemize}
Thus the $\mathsf{S}(t)$ is $p+1$-step determinable, then the propostion is right when $k^1 =p+1$.

As the propostion is right when $k^1 =1$, and we have if for $k^1=1,\ldots, p$ the propostion is right, then the propostion is right when $k^1=p+1$. Thus the propostion is right for any $k^1>0$.

Secondly, we have that if a \BCN\ satisfies the third observabbility, then there exists an input sequence $\mathsf{I}\in(\Delta_M)^{k^i}$ for some $k^i >0$, such that for any distinct states $\mathsf{s}^{x}(0)$, $\mathsf{s}^{y}(0) \in \Delta_N$, $h(\mathsf{s}^{x}(0))=h(\mathsf{s}^{y}(0))$ implies $(HF)^{k^i}_{\mathsf{s}^{x}(0)}(\mathsf{I})\neq (HF)^{k^i}_{\mathsf{s}^{y}(0)}(\mathsf{I})$. 

Therefore, we have the for every \[\mathsf{o}^{j}(0)\in \Delta_Q\] such that \[|\Ded\left(\Delta_N,\varepsilon, \mathsf{o}^{j}(0)\right)|> 0.\] There exists a $k^{j}\ge0$ such that $\Ded\left(\Delta_N,\varepsilon,\mathsf{o}^{j}(0)\right)$ is $k^{j}$-step determinable, and then the \BCN\ is online observable.
 \end{itemize}
\end{proof}

\end{appendices}
