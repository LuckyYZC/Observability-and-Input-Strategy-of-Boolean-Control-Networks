\begin{appendices}
\section{Proof}
\label{sec:pro}
\subsection{Proof of Lemma \ref{lemm:1}}
\begin{proof}
%$\mathsf{S}^{1}\subseteq \mathsf{S}^{2}$, 
If $\Ks(\mathsf{S}^{2})=0$, we have  \[\mathsf{S}^{1} = \mathsf{S}^{2}.\] Therefore, we have $\Ks(\mathsf{S}^{1})=0\le\Ks(\mathsf{S}^{2})$, thus we have the {\em Lemma \ref{lemm:1}} is right when $\Ks(\mathsf{S}^{2})=0$.
 
 If for $\Ks(\mathsf{S}^{2})=0,\ldots, k$ the {\em Lemma \ref{lemm:1}} is right. When $\Ks(\mathsf{S}^{2})=k+1$, we have for $\mathsf{S}^{2}$ there exists a $\mathsf{i}\in \mathcal{I}$ such that
 \begin{itemize}
 \item  $|\Ded\left(\mathsf{S}^{2},\mathsf{i},\varepsilon\right)|=|\mathsf{S}^{2}|$, and 
 \item  $\Ks(\Ded(\mathsf{S}^{2},\mathsf{i},\mathsf{o}))<(k+1)$, for each non-empty $\Ded(\mathsf{S}^{2},\mathsf{i},\mathsf{o})$.
 \end{itemize}
 and there is not any $\mathsf{i} \in \mathcal{I}$ satisfies that
  \begin{itemize}
 \item  $|\Ded\left(\mathsf{S}^{2},\mathsf{i},\varepsilon\right)|=|\mathsf{S}^2|$,
 \item  $\Ks(\Ded\left(\mathsf{S}^{2},\mathsf{i},\mathsf{o}\right))< k$ for each non-empty $\Ded\left(\mathsf{S}^{2},\mathsf{i},\mathsf{o}\right)$.
 \end{itemize} 
 Then we have for $\mathsf{S}^{1}$ there exists a $\mathsf{i}\in \mathcal{I}$ such that
 \begin{itemize}
 \item  $|\Ded\left(\mathsf{S}^{1},\mathsf{i},\varepsilon\right)|=|\mathsf{S}^{1}|$, and 
 \item  $\Ks(\Ded(\mathsf{S}^{1},\mathsf{i},\mathsf{o}))<(k'+1)\le (k+1)$, for each non-empty $\Ded(\mathsf{S}^{1},\mathsf{i},\mathsf{o})\subseteq \Ded(\mathsf{S}^{2},\mathsf{i},\mathsf{o})$.
 \end{itemize}
 and there is not any $\mathsf{i} \in \mathcal{I}$ satisfies that
  \begin{itemize}
 \item  $|\Ded\left(\mathsf{S}^{1},\mathsf{i},\varepsilon\right)|=|\mathsf{S}^1|$,
 \item  $\Ks(\Ded\left(\mathsf{S}^{1},\mathsf{i},\mathsf{o}\right))<k'\le k$ for each non-empty $\Ded\left(\mathsf{S}^{1},\mathsf{i},\mathsf{o}\right)$, $\Ded\left(\mathsf{S}^{1},\mathsf{i},\mathsf{o}\right)\subseteq \Ded\left(\mathsf{S}^{2},\mathsf{i},\mathsf{o}\right)$,
 \end{itemize} 
 Therefore we have if for $\Ks(\mathsf{S}^{2})=0,\ldots, k$ the {\em Lemma \ref{lemm:1}} is right, then when $\Ks(\mathsf{S}^{2})=k+1$ the {\em Lemma \ref{lemm:1}} is right too. 
And as the {\em Lemma \ref{lemm:1}} is right when $\Ks(\mathsf{S}^{2})=0$, we have the {\em Lemma \ref{lemm:1}} is right for every $\Ks(\mathsf{S}^{2})\ne \infty$.
 
\end{proof}
\subsection{Proof of Lemma \ref{lemm:2}}
\begin{proof}
For the non-empty $\mathsf{S}^{1}$ and $\mathsf{S}^{2}$, if $\mathsf{S}^{1}\subseteq\mathsf{S}^{2}$ and $\Ks(\mathsf{S}^{2})\ne\infty$.
For every $\mathsf{i}\in \Ri(\mathsf{S}^2)$, we have  $|\Ded\left(\mathsf{S}^2,\mathsf{i},\varepsilon\right)|=|\mathsf{S}^2|$, and 
for every $\mathsf{o} \in \mathcal{O}$, $\Ded(\mathsf{S}^2,\mathsf{i},\mathsf{o})\ne \emptyset$ implies $\Ks(\Ded(\mathsf{S}^2,\mathsf{i},\mathsf{o}))\ne \infty$. With {\em Lemma \ref{lemm:1}} we have $|\Ded\left(\mathsf{S}^1,\mathsf{i},\varepsilon\right)|=|\mathsf{S}^1|$, and 
for every $\mathsf{o} \in \mathcal{O}$, $\Ded(\mathsf{S}^1,\mathsf{i},\mathsf{o})\ne \emptyset$ implies $\Ks(\Ded(\mathsf{S}^1,\mathsf{i},\mathsf{o}))\le \Ks(\Ded(\mathsf{S}^2,\mathsf{i},\mathsf{o}))$, thus $\mathsf{i}\in \Ri(\mathsf{S}^1)$. Therefore, $\Ri(\mathsf{S}^2)\subseteq\Ri(\mathsf{S}^1)$.
\end{proof}




\subsection{Proof of Lemma \ref{lemm:4}}

\begin{proof}
Firsrly, we prove the propostion that for a set of possible state $\mathsf{S}(t)$, if there exists an input sequence $\mathsf{I}[t,t_k]\in\mathcal{I}^{[t,t_k]}$ for some $t_k >t$, such that for any distinct states $\mathsf{s}(t)$, $\mathsf{s}'(t) \in \mathsf{S}(t)$, $H^{[t,t_k]}(\mathsf{s}'(t),\mathsf{I}[t,t_k])\neq H^{[t,t_k]}(\mathsf{s}(t), \mathsf{I}[t,t_k])$, then $\Ks(\mathsf{S}(t))\ne\infty$.

\begin{itemize}
\item When $t_k=t+1$, for any two distinct states $\mathsf{s}(t)$, $\mathsf{s}'(t) \in \mathsf{S}(t)$, $H^{[t,t_k]}(\mathsf{s}'(t),\mathsf{I}[t,t_k])\neq H^{[t,t_k]}(\mathsf{s}(t), \mathsf{I}[t,t_k])$. Then we have for $\mathsf{S}(t)$,
 the input $\mathsf{i}(t)=\mathsf{I}[t,t_k]$ satisfies that
 \begin{itemize}
 \item  $|\Ded\left(\mathsf{S}(t),\mathsf{i}(t),\varepsilon\right)|=|\mathsf{S}(t)|$, and 
 \item  for every non-empty $\Ded(\mathsf{S}(t),\mathsf{i}(t),\mathsf{o}(t+1))$, the $\Ks(\Ded(\mathsf{S}(t),\mathsf{i}(t),\mathsf{o}(t+1)))=0$.
 \end{itemize}
Thus the $\Ks(\mathsf{S}(t))=1$, then the propostion is right when $t_k =t+1$.

\item If for $k=(t+1),\ldots, (t+n)$ the propostion is right. Then when $t_k=t+n+1$, for any distinct states $\mathsf{s}(t)$, $\mathsf{s}'(t) \in \mathsf{S}(t)$, $H^{[t,t_k]}(\mathsf{s}'(t),\mathsf{I}[t,t_k])\neq H^{[t,t_k]}(\mathsf{s}(t), \mathsf{I}[t,t_k])$. Then we have for $\mathsf{S}(t)$,
 there exists an input $\mathsf{i}(t)$ which is the first input of $\mathsf{I}[t,t_k]$, such that
 \begin{itemize}
\item  $|\Ded\left(\mathsf{S}(t),\mathsf{i}(t),\varepsilon\right)|=|\mathsf{S}(t)|$, and 
 \item  for every non-empty $\Ded(\mathsf{S}(t),\mathsf{i}(t),\mathsf{o}(t+1))$, $\Ks(\Ded(\mathsf{S}(t),\mathsf{i}(t),\mathsf{o}(t+1)))\ne \infty$.
 \end{itemize}
Thus the $\Ks(\mathsf{S}(t))\ne \infty$, then the propostion is right when $t_k =t+n+1$.

As the propostion is right when $t_k =t+1$, and we have if for $t_k=(t+1),\ldots, (t+n)$ the propostion is right, then the propostion is right when $t_k=t+n+1$. Thus the propostion is right for any $t_k>t$.

Secondly, we have that if a \BCN\ satisfies the {\bf Type-III} observability, then there exists an input sequence $\mathsf{I}[t]\in\mathcal{I}^{[t]}$ for some $t>0$, such that for any two distinct states $\mathsf{s}(0)$, $\mathsf{s}'(0) \in \mathcal{S}$, $H^{[0,t]}(\mathsf{s}'(0),\mathsf{I}[t])\neq H^{[0,t]}(\mathsf{s}(0), \mathsf{I}[t])$.

Therefore, we have the for every non-empty $\Ded(\mathcal{S}_\BB,\varepsilon, \mathsf{o})$, $\Ks(\Ded(\mathcal{S}_\BB,\varepsilon,\mathsf{o}))\ne \infty$, and then the \BCN\ is online observable.
 \end{itemize}
\end{proof}

\subsection{Proof of Lemma \ref{lemm:3}}
\begin{proof} Firsrly, we prove the propostion that for a set of possible state $\mathsf{S}(t)$ if $\Ks(\mathsf{S}(t))\ne\infty$, then for every state \State$(t)\in \mathsf{S}(t)$, there exists an input sequence $\mathsf{I}[t,t_k]\in\mathcal{I}^{[t,t_k]}$ for some $t_k >t$ such that for every $\mathsf{s}'(t)\in \mathsf{S}(t)$, $\mathsf{s}'(t)\neq \mathsf{s}(t)$, $H^{[t,t_k]}(\mathsf{s}'(t),\mathsf{I}[t,t_k])\neq H^{[t,t_k]}(\mathsf{s}(t), \mathsf{I}[t,t_k])$.
\begin{itemize}
\item When $\Ks(\mathsf{S}(t))=0$, we have $|\mathsf{S}(t)|=1$, then for every \State$(t)$$\in \mathsf{S}(t)$ there does not exist any $\mathsf{s}'(t)\in \mathsf{S}(t)$ that $\mathsf{s}(t)\neq \mathsf{s}(t)$. Therefore, we have that for any $t_k >t$, for every input sequence $\mathsf{I}[t,t_k]\in\mathcal{I}^{[t,t_k]}$ the propostion is right. 
\item If for $\Ks(\mathsf{S}(t))=0,\ldots, n$ the propostion is right. When $\Ks(\mathsf{S}(t))=n+1$, we have for $\mathsf{S}(t)$ there exists a $\mathsf{i}(t)\in \mathcal{I}$ such that
 \begin{itemize}
 \item  $|\Ded\left(\mathsf{S}(t),\mathsf{i}(t),\varepsilon\right)|=|\mathsf{S}(t)|$, and 
 \item  for every non-empty $\Ded(\mathsf{S}(t),\mathsf{i}(t),\mathsf{o}(t+1))$, $\Ks(\Ded(\mathsf{S}(t),\mathsf{i}(t),\mathsf{o}(t+1)))<(n+1)$.
 \end{itemize}
 Then we have for every \State$(t)$$\in \mathsf{S}(t)$, there exists an $\mathsf{I}[t,t_k]\in\mathcal{I}^{[t,t_k]}$ for some $t_k >t$, such that for all $\mathsf{s}'(t)\in \mathsf{S}(t)$, $\mathsf{s}'(t)\neq \mathsf{s}(t)$ implies $H^{[t,t_k]}(\mathsf{s}'(t),\mathsf{I}[t,t_k])\neq H^{[t,t_k]}(\mathsf{s}(t), \mathsf{I}[t,t_k])$, where the input sequence $\mathsf{I}[k]=\mathsf{i}(t)\mathsf{I}[t+1,t_k]$. That the input sequence $\mathsf{I}[t+1,t_k]$ satisfies %the following properties,
  %for all states \[\mathsf{s}'(t+1)\in \Ded(\mathsf{S}(t),\mathsf{i}^{i}(t),h(\mathsf{s}(t+1))),\]\[\mathsf{s}'(t+1)\neq \mathsf{s}(t+1)\] implies 
  \[H^{[t+1,t_k]}(\mathsf{s}'(t+1),\mathsf{I}[t+1,t_k])\neq H^{[t+1,t_k]}(\mathsf{s}(t+1), \mathsf{I}[t+1,t_k]),\] where $\mathsf{s}(t+1)=\sigma(\mathsf{s}(t),\mathsf{i}(t))$ and $\mathsf{s}'(t+1)=\sigma(\mathsf{s}'(t),\mathsf{i}(t))$. Then we have the propostion is right when $\Ks(\mathsf{S}(t))=n+1$. 

\end{itemize}
As the propostion is right when $\Ks(\mathsf{S}(t))=0$, and we have if for $\Ks(\mathsf{S}(t))=0,\ldots, n$ the propostion is right, then the propostion is right when $\Ks(\mathsf{S}(t))=n+1$. Thus the propostion is right for every $\Ks(\mathsf{S}(t))\ne\infty$.

Secondly, we have that if a \BCN\ is online observable,
then for every  non-empty $\Ded\left(\mathcal{S}_\BB,\varepsilon, \mathsf{o}\right)$, $\Ks(\Ded\left(\mathcal{S}_\BB,\varepsilon, \mathsf{o}\right))\ne \infty$

Therefore, we have for every initial state \State$(0)$$\in \mathcal{S}$, there exists an input sequence  $\mathsf{I}[t]\in\mathcal{I}^{[t]}$ for some $t>0$, such that for all states $\mathsf{s}'(0)\neq \mathsf{s}(0)$, $H^{[0,t]}(\mathsf{s}'(0),\mathsf{I}[t])\neq H^{[0,t]}(\mathsf{s}(0), \mathsf{I}[t])$. Thus the \BCN\ satisfies the  {\bf Type-I} observability if it is online observable.
\end{proof}
\end{appendices}
