
% *** PDF, URL AND HYPERLINK PACKAGES ***
%
%\usepackage{url}
% url.sty was written by Donald Arseneau. It provides better support for
% handling and breaking URLs. url.sty is already installed on most LaTeX
% systems. The latest version and documentation can be obtained at:
% http://www.ctan.org/pkg/url
% Basically, \url{my_url_here}.

\newtheorem{example}{Example}
\newtheorem{remark}{Remark}
\newtheorem{definition}{Definition}
\newtheorem{theorem}{Theorem}
\newtheorem{lemma}{Lemma}
\newtheorem{proof}{Proof}
\newtheorem{proposition}{\bf Proposition}
\newtheorem{problem}{Problem}


\newcommand{\tl}[1]{\textcolor{blue} {TL: #1 :TL} }
\newcommand{\ly}[1]{\textcolor{red} {LY: #1 :LY} }
\newcommand{\gs}[1]{\textcolor{green} {GS: #1 :GS} }

% *** Do not adjust lengths that control margins, column widths, etc. ***
% *** Do not use packages that alter fonts (such as pslatex).         ***
% There should be no need to do such things with IEEEtran.cls V1.6 and later.
% (Unless specifically asked to do so by the journal or conference you plan
% to submit to, of course. )

\def \BN {{\em BN}}
\def \BNs {{\em BNs}}
\def \BCN {{\em BCN}}
\def \BCNs {{\em BCNs}}
\def \STP {{\em STP}}
\def \Pe {{\tt Pe}}
\def \Pv {{\tt Pv}}
\def \Spe {{\tt Spe}}
\def \Ce {{\tt Ce}}
\def \Cv {{\tt Cv}}
\def \Ks {\Gamma}
\def \Ri {\psi}
\def \Lce {{\tt Lce}}
\def \Ded {\zeta}
\def \BB {{\mathcal{B}}}
\newcommand \Input{{{$\mathsf{i}$}}}
\newcommand \State {{{$\mathsf{s}$}}}
\newcommand\Output {{{$\mathsf{o}$}}}
\newcommand  \Ustate {{{$\mathsf{S}$}}}

\newcommand{\rev}[1]{{\color{red}{#1}}}



%%%%%%%%%%%%%%%%%%%%%%%%%%%%
%for comments
\newcommand\JP[1]{\textcolor{magenta}{#1}}
%%%%%%%%%%%%%%%%%%%%%%%%%%%%