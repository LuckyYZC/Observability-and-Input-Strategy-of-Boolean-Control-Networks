% !Mode\dots ``TeX:UTF-8''
% !TEX root = ../root.tex
\section{Conclusions}
\label{sec:con}
In this paper, firstly we proposed the online observability of {\em BCNs} and define its mathematical form. Secondly we use the super tree and directed graph to determine the online observability. After introduced the ways to determine the online observability we present some applications of the online observability of {\em BCNs} and talk about some advantages of it. %use it to try to find the shortest path and avoid entering critical states when we determining the initial state of {\em BCNs}. 

But even we use the directed graph, it is still hard to determine the  the online observability of a \BCN\ with a large number of nodes. Therefore, in the future we will try to separate the {\em BCNs}, and then determine their online observability respectively. Furthermore, we also want to try to use some knowledge about formal methods to earn scalability for {\em BCNs}. In addition to the theoretical aspect, the realistic application is also very important. Hence we will also try to find some realistic example which can be modeled by {\em BCNs}. So that we can research these realistic examples well and determine the online observability their models for better performance.