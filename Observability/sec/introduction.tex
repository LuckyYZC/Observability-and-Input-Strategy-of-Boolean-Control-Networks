% !Mode\dots ``TeX:UTF-8''
% !TEX root = ../root.tex
\section{INTRODUCTION}

In 1960s, Nobel Prize winners Jacob and Monod found that  ``Any cell contains a number of `regulatory' genes that act as switches and can turn one another on and off. If genes can turn one another on and off, then you can have genetic circuits.'' \cite{Waldrop1992Complexity,cheng2009controllability}. Inspired by these Boolean-type actions in genetic circuits, the Boolean networks ({\em BNs}) is firstly proposed by Kauffman \cite{Kauffman1968Metabolic} for modeling nonlinear and complex biological systems. Some general descriptions of the \BNs\ and its applications to biological systems can be found in Kauffman. Since then research interests in  \BNs\ have been motivated by the large number of natural and artificial systems. These natural and artificial systems describing variables display only two distinct configurations, and hence these describing variables take only two values, i.e., $\{0,1\}$  \cite{Akutsu2000Inferring, Shmulevich2002From, Faur2006Dynamical,Green2007The,Lou2010Multi,Fornasini2013Observability}.


When extenal regulation or perturbation is considered, \BNs\ are naturally extended to Boolean control networks ({\em BCNs}) \cite{Ideker2001A}. The study on control-theoretic problems of \BCNs\ can date back to 2007 \cite{Akutsu2007Control}. The work above also proves that the problem of determining the controllability of \BCNs\ is {\em NP}-hard in the number of nodes. Furthermore, it points out that ``One of the major goals of systems biology is to develop a control theory for complex biological systems.'' Since then, the study on control-theoretic problems in the areas of \BNs\ and \BCNs\ has drawn great attention \cite{cheng2009controllability, Zhao2010Input, Cheng2011Identification, Cheng2011Analysis,Fornasini2013Observability}. \rev{Besides controllability, observability is an another  basic control-theoretic problems and it also attract many attentions}.  \rev{Among the studies, \emph{semi-tensor product} (STP) of matrices is one of useful tool to deal with  both \BNs\ and \BCNs\  related problems \cite{cheng2009controllability}.}  \rev{Moreover,  \cite{cheng2009controllability} gives} equivalent conditions for controllability of \BCNs\ and observability of controllable {\em BCNs}. To date, there are four types of observability have been proposed. 

\begin{enumerate}
	\item The first type of observability proposed in 2009 \cite{cheng2009controllability} means that every initial state can be determined by an input sequence.
	
	\item 
	The second observability proposed in 2010 \cite{Zhao2010Input} stands that for every two distinct initial states, there exists an input sequence which can distinguish them, and this observability is determined in \cite{Li2015Controllability}.
	
	\item The third observability proposed in 2011 \cite{Cheng2011Identification} states that there is an input sequence that determines the initial state.
	
	\item  The fourth observability proposed in 2013 \cite{Fornasini2013Observability} is essentially the observability of linear control systems, i.e., every sufficient long input sequence can determine the initial state.
\end{enumerate}
 

%\tl{can you state the four types observability clearly and formally here?}

%\rev{****input s equence***}

In above mentioned definitions an input is not the value of an input-node, but it represents the values of all input-nodes of the \BCN\ on a time step. Therefore, an input is a vector of the values of all input-nodes of the \BCN\ on a time step. An input sequence consists of several inputs in sequential time steps.
     \rev{A output is a vector of the values of all output-nodes of the \BCN\ on a time step. In every step, there is a pair of input and output. A output sequence also consists of several outputs in sequential time steps. In the following, we will list the inform definition of four offline  observabilities and they formal definition of four observabilities in the following pages.}
 
\rev{The four  types of observability  provide many nice properties and they can be used into some useful applications.} However, all of four  types of observability of \BCNs\ are offline observabilities which means that they can't adjust the input sequence by observing the output sequence in the process of determining the initial state of {\em BCNs}. Therefore, we propose the online observability that we can determine the initial state of \BCNs\ dynamically. \rev{In other words,}  the online observability decides the input sequence in each time step by observes the out sequence. \rev{ In the  online observability, we   infer the possible  initial states set by observe the  first $k$ time steps output of {\em BCN}. Through the  possible  initial states, we can choose one of input to refine the possible initial states set in $k+1$ time step. Repeat above procedure until the cardinality of initial states set is one}. We call this process is a dynamic process. 

\subsubsection*{Contribution}
Firstly, we give the formal definition of online observability of {\em BCNs}. Secondly, we provide two algorithms to determine the online observability for {\em BCNs}. Finally, we introduce some applications of the online observability of {\em BCNs} and advantages of online observability comparing with offline observabilities. \rev{No important points}%\rev{***Compare with offline observabilities****} 
\subsubsection*{Structure}
The remainder of this paper is organized as follows. {\em Section II} introduces necessary preliminaries about {\em BCNs}, algebraic forms of \BCNs\ and the four existing kinds of observability of {\em BCNs}. {\em Section III} presents the definition of deduce function, $K$ steps deterministic and online observability of {\em BCNs}. {\em Section IV} presents how to determine the online observability of \BCNs\ by super tree and directed graph. {\em Section V} talks about some applications of the online observability of {\em BCNs}. We also compare the online observability with offline observabilities in this section. {\em Section VI} ends up  with the introduction of some future works.

%\tl{I will try to rewrite the intro.}

%==============================================================================================================